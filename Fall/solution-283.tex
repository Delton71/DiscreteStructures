\begin{task}{283}
Ориентированный граф называется \textit{сильно связным}, если для любых двух его вершин \(s\) и \(t\) существует как ориентированный путь из \(s\) в \(t\), так и ориентированный путь из \(t\) в \(s\). Докажите, что в любом сильно связном турнире существует ориентированный гамильтонов цикл.
\end{task}

Будем рассматривать турниры, обозначив \(V\)~--- множество его вершин, \(E\)~--- множество его дуг и \(|V| \geqslant 3\). Докажем, используя метод математической индукции по числу вершин в цикле.

Для начала покажем, что в сильно связном турнире найдется цикл длины \(3\).

Выберем любую вершину \(v\). Так как турнир сильно связный, найдутся и входящие, и выходящие дуги~--- разделим их на два непересекающихся множества \(I\) и \(O\). Более формально:
\begin{align*}
    I &= \{i \in V\:\mid\:(i, v) \in E \},\\
    O &= \{o \in V\:\mid\:(v, o) \in E \}.
\end{align*}

Так как турнир сильно связный:
\begin{enumerate}
    \item \(I\) и \(O\) не пустые;
    
    \item Существует связывающее \(O\) и \(I\) дуга:
    \begin{equation*}
        \exists (i_1 \in I, o_1 \in O)\:\mid\:(o_1, i_1) \in E,
    \end{equation*}
    иначе не было бы пути из одного множества в другое.
\end{enumerate}

Наш искомый цикл (\(v - o_1 - i_1 - v\)) и его длины равна трем.

Теперь предположим, что в турнире есть простой цикл длины \(k < |V|\). Пусть это будет
\begin{equation*}
    C_k = (v_1 \rightarrow v_2 \rightarrow \ldots \rightarrow v_k \rightarrow v_1).
\end{equation*}

Так длины цикла меньше количества вершин турнира, \(\exists v_0 \in V, v_0 \notin C_k\). Возможны две ситуации:
\begin{enumerate}
    \item Существуют вершины цикла \(v_i, v_j\):
    \begin{equation*}
        \exists (v_i, v_0), (v_0, v_j) \in E;
    \end{equation*}
    
    \item Таких вершин \(v_i, v_j\) не нашлось.
\end{enumerate}

В первом случае мы обозначим за \(v_j\) \textbf{первую} вершину при обходе \(C_k\), для которой \(\exists (v_o, v_j) \in E\). Тогда вершина \(v_{j-1}\) такова, что \(\exists (v_{j-1}, v_o) \in E\). Она существует, так как существует входящая в \(v_0\) из цикла дуга. Тогда наш искомый цикл длины \(k + 1\):
\begin{equation*}
    (v_1 \rightarrow v_2 \rightarrow \ldots \rightarrow v_{j-1} \rightarrow v_0 \rightarrow v_j \rightarrow \ldots \rightarrow v_k \rightarrow v_1).
\end{equation*}

Заметим также, что, если в \(C_k\) не входит лишь одна вершина, образуется первая ситуация, иначе дуги \(v_0\) только входящие или только исходящие, что противоречит условию сильной связности.

Во втором случае мы разделяем вершины, не принадлежащие циклу \(C_k\) на два непересекающихся множества входящих в его вершины и выходящих из них:
\begin{align*}
    I_{C_k} &= \{x \in V \wedge x \notin C_k\:\mid\:(\forall v_i \in C_k)\:(x, v_i) \in E \},\\
    O_{C_k} &= \{y \in V \wedge y \notin C_k\:\mid\:(\forall v_i \in C_k)\:(v_i, y) \in E \}.
\end{align*}

Так как турнир сильно связный:
\begin{enumerate}
    \item \(I_{C_k}\) и \(O_{C_k}\) не пустые;
    
    \item Существует связывающее \(O_{C_k}\) и \(I_{C_k}\) дуга:
    \begin{equation*}
        \exists (i_2 \in I_{C_k}, o_2 \in O_{C_k})\:\mid\:(o_2, i_2) \in E,
    \end{equation*}
    иначе не было бы пути из одного множества в другое.
\end{enumerate}

Соответственно, наш искомый цикл длины \(k + 1\):
\begin{equation*}
    (v_1 \rightarrow o_2 \rightarrow i_2 \rightarrow v_3 \rightarrow \ldots \rightarrow v_k \rightarrow v_1).
\end{equation*}

В каждом из двух случаев находится цикл длины \(k + 1\), значит, по методу математической индукции в сильно связном турнире \(|V| \geqslant 3\) найдутся циклы длины (\(3, 4, \ldots, |V|\)), значит, найдется и ориентированный гамильтонов цикл.
