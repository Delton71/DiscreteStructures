\begin{task}{468}
Найдите асимптотику количества различных раскрасок (необязательно правильных) графа на рисунке в \(k\) цветов  при \(k\to\infty\). Раскраски считаются одинаковыми, если они переходят друг в друга при некотором автоморфизме графа. Используйте теорему Редфилда\,--\,Пойи.
\end{task}

\begin{solution}

% Авточекер очень много ругается на tikz: он видит в "\node (1)" неправильное построение уравнения... 
% Бедное сочетание "Редфилда\,--\,Пойи", написание которого, между прочим, взято из условия, вообще "раздербанил" на несколько замечаний
% Синтаксически верное оформление предложения в строке 40 (двоеточие, тире) воспринял как правило сокращения порядковых числительных...
% В общем, авточекер прогнан, но он выворачивает действительность в обратную сторону. К сожалению, с библиотекой tikz он вообще не дружит. 

% Остальные замечания авточекера я исправил

Пронумеруем вершины графа и заметим симметрию при поворотах графа на \(90\) градусов:
\begin{figure}[H]
    \centering
    \tikz {
        \node (1) [circle, draw] at (2, 3) {1};
        \node (2) [circle, draw] at (3, 3) {2};
        \node (3) [circle, draw] at (3, 2) {3};
        \node (4) [circle, draw] at (2, 2) {4};
        \node (5) [circle, draw] at (2, 4) {5};
        \node (6) [circle, draw] at (3, 4) {6};
        \node (7) [circle, draw] at (4, 3) {7};
        \node (8) [circle, draw] at (4, 2) {8};
        \node (9) [circle, draw] at (3, 1) {9};
        \node (10) [circle, draw, scale=0.8] at (2, 1) {10};
        \node (11) [circle, draw, scale=0.8] at (1, 2) {11};
        \node (12) [circle, draw, scale=0.8] at (1, 3) {12};
        \node (sym1) at (2.5, 5) {};
        \node (sym2) at (2.5, 0) {};
        \node (sym3) at (0, 2.5) {};
        \node (sym4) at (5, 2.5) {};        
        
        \graph {
            (1) -- (2), (2) -- (3), (3) -- (4), (4) -- (1), (1) -- (3), (2) -- (4),
            (1) -- (5), (5) -- (6), (6) -- (2),
            (2) -- (7), (7) -- (8), (8) -- (3),
            (3) -- (9), (9) -- (10), (10) -- (4),
            (4) -- (11), (11) -- (12), (12) -- (1),

            (sym1) -- [dotted] (sym2), (sym3) -- [dotted] (sym4)
        };
    }
\end{figure}

В группе автоморфизмов, помимо тождественной перестановки, существуют три перестановки с поворотами вокруг центра тяжести \((1, 2, 3, 4)\) по часовой стрелке: \((1, 2, 3, 4)(5, 7, 9, 11)(6, 8, 10, 12)\). А также <<взгляд с обратной стороны>>: \\ \((1, 2)(3, 4)(5, 6)(7, 12)(8, 11)(9, 10)\),~--- и те же перестановки с поворотом по часовой стрелке на \(90\) градусов.

Получаем \(8\) автоморфизмов:
\begin{figure}[H]
    \centering
    \begin{subfigure}[b]{0.4\linewidth}
        \centering
        \tikz {
            \node (1) [circle, draw] at (2, 3) {1};
            \node (2) [circle, draw] at (3, 3) {2};
            \node (3) [circle, draw] at (3, 2) {3};
            \node (4) [circle, draw] at (2, 2) {4};
            \node (5) [circle, draw] at (2, 4) {5};
            \node (6) [circle, draw] at (3, 4) {6};
            \node (7) [circle, draw] at (4, 3) {7};
            \node (8) [circle, draw] at (4, 2) {8};
            \node (9) [circle, draw] at (3, 1) {9};
            \node (10) [circle, draw, scale=0.8] at (2, 1) {10};
            \node (11) [circle, draw, scale=0.8] at (1, 2) {11};
            \node (12) [circle, draw, scale=0.8] at (1, 3) {12};
            \node (sym1) at (2.5, 5) {};
            \node (sym2) at (2.5, 0) {};
            \node (sym3) at (0, 2.5) {};
            \node (sym4) at (5, 2.5) {};        
            
            \graph {
                (1) -- (2), (2) -- (3), (3) -- (4), (4) -- (1), (1) -- (3), (2) -- (4),
                (1) -- (5), (5) -- (6), (6) -- (2),
                (2) -- (7), (7) -- (8), (8) -- (3),
                (3) -- (9), (9) -- (10), (10) -- (4),
                (4) -- (11), (11) -- (12), (12) -- (1),
    
                (sym1) -- [dotted] (sym2), (sym3) -- [dotted] (sym4)
            };
        }
    \end{subfigure}
    \begin{subfigure}[b]{0.4\linewidth} 
        \centering
        \tikz {
            \node (1) [circle, draw] at (3, 3) {1};
            \node (2) [circle, draw] at (3, 2) {2};
            \node (3) [circle, draw] at (2, 2) {3};
            \node (4) [circle, draw] at (2, 3) {4};
            \node (5) [circle, draw] at (4, 3) {5};
            \node (6) [circle, draw] at (4, 2) {6};
            \node (7) [circle, draw] at (3, 1) {7};
            \node (8) [circle, draw] at (2, 1) {8};
            \node (9) [circle, draw] at (1, 2) {9};
            \node (10) [circle, draw, scale=0.8] at (1, 3) {10};
            \node (11) [circle, draw, scale=0.8] at (2, 4) {11};
            \node (12) [circle, draw, scale=0.8] at (3, 4) {12};
            \node (sym1) at (2.5, 5) {};
            \node (sym2) at (2.5, 0) {};
            \node (sym3) at (0, 2.5) {};
            \node (sym4) at (5, 2.5) {};        
            
            \graph {
                (1) -- (2), (2) -- (3), (3) -- (4), (4) -- (1), (1) -- (3), (2) -- (4),
                (1) -- (5), (5) -- (6), (6) -- (2),
                (2) -- (7), (7) -- (8), (8) -- (3),
                (3) -- (9), (9) -- (10), (10) -- (4),
                (4) -- (11), (11) -- (12), (12) -- (1),
    
                (sym1) -- [dotted] (sym2), (sym3) -- [dotted] (sym4)
            };
        }
    \end{subfigure}
    \begin{subfigure}[b]{0.4\linewidth} 
        \centering
        \tikz {
            \node (1) [circle, draw] at (3, 2) {1};
            \node (2) [circle, draw] at (2, 2) {2};
            \node (3) [circle, draw] at (2, 3) {3};
            \node (4) [circle, draw] at (3, 3) {4};
            \node (5) [circle, draw] at (3, 1) {5};
            \node (6) [circle, draw] at (2, 1) {6};
            \node (7) [circle, draw] at (1, 2) {7};
            \node (8) [circle, draw] at (1, 3) {8};
            \node (9) [circle, draw] at (2, 4) {9};
            \node (10) [circle, draw, scale=0.8] at (3, 4) {10};
            \node (11) [circle, draw, scale=0.8] at (4, 3) {11};
            \node (12) [circle, draw, scale=0.8] at (4, 2) {12};
            \node (sym1) at (2.5, 5) {};
            \node (sym2) at (2.5, 0) {};
            \node (sym3) at (0, 2.5) {};
            \node (sym4) at (5, 2.5) {};        
            
            \graph {
                (1) -- (2), (2) -- (3), (3) -- (4), (4) -- (1), (1) -- (3), (2) -- (4),
                (1) -- (5), (5) -- (6), (6) -- (2),
                (2) -- (7), (7) -- (8), (8) -- (3),
                (3) -- (9), (9) -- (10), (10) -- (4),
                (4) -- (11), (11) -- (12), (12) -- (1),
    
                (sym1) -- [dotted] (sym2), (sym3) -- [dotted] (sym4)
            };
        }
    \end{subfigure}
    \begin{subfigure}[b]{0.4\linewidth} 
        \centering
        \tikz {
            \node (1) [circle, draw] at (2, 2) {1};
            \node (2) [circle, draw] at (2, 3) {2};
            \node (3) [circle, draw] at (3, 3) {3};
            \node (4) [circle, draw] at (3, 2) {4};
            \node (5) [circle, draw] at (1, 2) {5};
            \node (6) [circle, draw] at (1, 3) {6};
            \node (7) [circle, draw] at (2, 4) {7};
            \node (8) [circle, draw] at (3, 4) {8};
            \node (9) [circle, draw] at (4, 3) {9};
            \node (10) [circle, draw, scale=0.8] at (4, 2) {10};
            \node (11) [circle, draw, scale=0.8] at (3, 1) {11};
            \node (12) [circle, draw, scale=0.8] at (2, 1) {12};
            \node (sym1) at (2.5, 5) {};
            \node (sym2) at (2.5, 0) {};
            \node (sym3) at (0, 2.5) {};
            \node (sym4) at (5, 2.5) {};        
            
            \graph {
                (1) -- (2), (2) -- (3), (3) -- (4), (4) -- (1), (1) -- (3), (2) -- (4),
                (1) -- (5), (5) -- (6), (6) -- (2),
                (2) -- (7), (7) -- (8), (8) -- (3),
                (3) -- (9), (9) -- (10), (10) -- (4),
                (4) -- (11), (11) -- (12), (12) -- (1),
    
                (sym1) -- [dotted] (sym2), (sym3) -- [dotted] (sym4)
            };
        }
    \end{subfigure}
    \caption{Повороты на 90 градусов.}
\end{figure}

\begin{figure}[H]
    \centering
    \begin{subfigure}[b]{0.4\linewidth}
        \centering
        \tikz {
            \node (1) [circle, draw] at (3, 3) {1};
            \node (2) [circle, draw] at (2, 3) {2};
            \node (3) [circle, draw] at (2, 2) {3};
            \node (4) [circle, draw] at (3, 2) {4};
            \node (5) [circle, draw] at (3, 4) {5};
            \node (6) [circle, draw] at (2, 4) {6};
            \node (7) [circle, draw] at (1, 3) {7};
            \node (8) [circle, draw] at (1, 2) {8};
            \node (9) [circle, draw] at (2, 1) {9};
            \node (10) [circle, draw, scale=0.8] at (3, 1) {10};
            \node (11) [circle, draw, scale=0.8] at (4, 2) {11};
            \node (12) [circle, draw, scale=0.8] at (4, 3) {12};
            \node (sym1) at (2.5, 5) {};
            \node (sym2) at (2.5, 0) {};
            \node (sym3) at (0, 2.5) {};
            \node (sym4) at (5, 2.5) {};        
            
            \graph {
                (1) -- (2), (2) -- (3), (3) -- (4), (4) -- (1), (1) -- (3), (2) -- (4),
                (1) -- (5), (5) -- (6), (6) -- (2),
                (2) -- (7), (7) -- (8), (8) -- (3),
                (3) -- (9), (9) -- (10), (10) -- (4),
                (4) -- (11), (11) -- (12), (12) -- (1),
    
                (sym1) -- [dotted] (sym2), (sym3) -- [dotted] (sym4)
            };
        }
    \end{subfigure}
    \begin{subfigure}[b]{0.4\linewidth}
        \centering
        \tikz {
            \node (1) [circle, draw] at (3, 2) {1};
            \node (2) [circle, draw] at (3, 3) {2};
            \node (3) [circle, draw] at (2, 3) {3};
            \node (4) [circle, draw] at (2, 2) {4};
            \node (5) [circle, draw] at (4, 2) {5};
            \node (6) [circle, draw] at (4, 3) {6};
            \node (7) [circle, draw] at (3, 4) {7};
            \node (8) [circle, draw] at (2, 4) {8};
            \node (9) [circle, draw] at (1, 3) {9};
            \node (10) [circle, draw, scale=0.8] at (1, 2) {10};
            \node (11) [circle, draw, scale=0.8] at (2, 1) {11};
            \node (12) [circle, draw, scale=0.8] at (3, 1) {12};
            \node (sym1) at (2.5, 5) {};
            \node (sym2) at (2.5, 0) {};
            \node (sym3) at (0, 2.5) {};
            \node (sym4) at (5, 2.5) {};        
            
            \graph {
                (1) -- (2), (2) -- (3), (3) -- (4), (4) -- (1), (1) -- (3), (2) -- (4),
                (1) -- (5), (5) -- (6), (6) -- (2),
                (2) -- (7), (7) -- (8), (8) -- (3),
                (3) -- (9), (9) -- (10), (10) -- (4),
                (4) -- (11), (11) -- (12), (12) -- (1),
    
                (sym1) -- [dotted] (sym2), (sym3) -- [dotted] (sym4)
            };
        }
    \end{subfigure}
    \begin{subfigure}[b]{0.4\linewidth}
        \centering
        \tikz {
            \node (1) [circle, draw] at (2, 2) {1};
            \node (2) [circle, draw] at (3, 2) {2};
            \node (3) [circle, draw] at (3, 3) {3};
            \node (4) [circle, draw] at (2, 3) {4};
            \node (5) [circle, draw] at (2, 1) {5};
            \node (6) [circle, draw] at (3, 1) {6};
            \node (7) [circle, draw] at (4, 2) {7};
            \node (8) [circle, draw] at (4, 3) {8};
            \node (9) [circle, draw] at (3, 4) {9};
            \node (10) [circle, draw, scale=0.8] at (2, 4) {10};
            \node (11) [circle, draw, scale=0.8] at (1, 3) {11};
            \node (12) [circle, draw, scale=0.8] at (1, 2) {12};
            \node (sym1) at (2.5, 5) {};
            \node (sym2) at (2.5, 0) {};
            \node (sym3) at (0, 2.5) {};
            \node (sym4) at (5, 2.5) {};        
            
            \graph {
                (1) -- (2), (2) -- (3), (3) -- (4), (4) -- (1), (1) -- (3), (2) -- (4),
                (1) -- (5), (5) -- (6), (6) -- (2),
                (2) -- (7), (7) -- (8), (8) -- (3),
                (3) -- (9), (9) -- (10), (10) -- (4),
                (4) -- (11), (11) -- (12), (12) -- (1),
    
                (sym1) -- [dotted] (sym2), (sym3) -- [dotted] (sym4)
            };
        }
    \end{subfigure}
    \begin{subfigure}[b]{0.4\linewidth}
        \centering
        \tikz {
            \node (1) [circle, draw] at (2, 3) {1};
            \node (2) [circle, draw] at (2, 2) {2};
            \node (3) [circle, draw] at (3, 2) {3};
            \node (4) [circle, draw] at (3, 3) {4};
            \node (5) [circle, draw] at (1, 3) {5};
            \node (6) [circle, draw] at (1, 2) {6};
            \node (7) [circle, draw] at (2, 1) {7};
            \node (8) [circle, draw] at (3, 1) {8};
            \node (9) [circle, draw] at (4, 2) {9};
            \node (10) [circle, draw, scale=0.8] at (4, 3) {10};
            \node (11) [circle, draw, scale=0.8] at (3, 4) {11};
            \node (12) [circle, draw, scale=0.8] at (2, 4) {12};
            \node (sym1) at (2.5, 5) {};
            \node (sym2) at (2.5, 0) {};
            \node (sym3) at (0, 2.5) {};
            \node (sym4) at (5, 2.5) {};        
            
            \graph {
                (1) -- (2), (2) -- (3), (3) -- (4), (4) -- (1), (1) -- (3), (2) -- (4),
                (1) -- (5), (5) -- (6), (6) -- (2),
                (2) -- (7), (7) -- (8), (8) -- (3),
                (3) -- (9), (9) -- (10), (10) -- (4),
                (4) -- (11), (11) -- (12), (12) -- (1),
    
                (sym1) -- [dotted] (sym2), (sym3) -- [dotted] (sym4)
            };
        }
    \end{subfigure}
    \caption{Повороты на 90 градусов, <<взгляд с обратной стороны>>.}
\end{figure}

По теореме Редфилда\,--\,Пойи, асимптотика количества различных раскрасок графа на \(n\) вершинах \(k\) цветов при \(k\to\infty\) есть:
\begin{equation*}
    \frac{k^n}{\left\lvert Aut \right\lvert} = \frac{k^{12}}{8}.
\end{equation*}

\textbf{Ответ: } \(\displaystyle \frac{k^{12}}{8}\).

\end{solution}
