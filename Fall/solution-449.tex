\begin{task}{449}
Пусть дана произвольная группа \(\mathbb{G}\) порядка \(\lvert\mathbb{G}\rvert = p^k,\, p\in\mathbb{P}\), докажите, что в \(\mathbb{G}\) имеется ровно \(p^m\) элементов \(a_i\) (для некоторого целого \(m>0\)), для каждого из которых его класс сопряжённости удовлетворяют равенству \(\left\lvert a_i^{\mathbb{G}}\right\rvert = 1\). В процессе решения кроме прочего вам может потребоваться доказать, что такое множество элементов образует подгруппу в \(\mathbb{G}\), также не помешает одно из следствий \emph{теоремы Лагранжа}.
\end{task}

\begin{solution}

\begin{enumerate}
    \item Если \(\left\lvert a^{\mathbb{G}}\right\rvert = 1\), то \(\forall g \in \mathbb{G}\:gag^{-1}=a \Leftrightarrow ag=ga\). Значит, \(a \in Z(\mathbb{G})\), а нам нужно доказать, что \(\left\lvert Z(\mathbb{G})\right\rvert = p^m\).

    \item Сопряжение~--- это отношение эквивалентности, разбивающее группу на непересекающиеся классы сопряженности, причем по теореме Лагранжа:
    \begin{equation*}
        \left\lvert a^{\mathbb{G}}\right\rvert = \frac{\left\lvert \mathbb{G}\right\rvert}{\left\lvert C_{\mathbb{G}}(a)\right\rvert},
    \end{equation*}
    другими словами, порядок класса сопряженности делит порядок группы.
    
    \item По прошлым пунктам мы можем утверждать, что:
    \begin{equation*}
        \left\lvert \mathbb{G}\right\rvert = \left\lvert Z(\mathbb{G})\right\rvert + \sum_i \left\lvert b_i^{\mathbb{G}}\right\rvert.
    \end{equation*}

    Сумма содержит не вошедшие в \(Z(\mathbb{G})\) элементы \(b_i\), то есть для каждого из них \(\left\lvert b_i^{\mathbb{G}}\right\rvert > 1\). А так как \(\lvert\mathbb{G}\rvert = p^k\) и порядок класса сопряженности делит порядок группы, получаем:
    \begin{equation*}
        \left\lvert \mathbb{G}\right\rvert = p^k = \left\lvert Z(\mathbb{G})\right\rvert + \sum_i p^{k_i}.
    \end{equation*}

    Получаем, что \(\left\lvert Z(\mathbb{G})\right\rvert\:\vdots\:p \Leftrightarrow \left\lvert Z(\mathbb{G})\right\rvert = p^m\), что и требовалось доказать.

\end{enumerate}

\end{solution}
