\begin{task}{461}
Найдите сумму количеств неупорядоченных разбиений натуральных чисел \(2\leqslant n\leqslant 57\), таких, что каждое разбиение состоит не более, чем из \(10\) слагаемых и максимальное слагаемое в каждом из которых не больше \(6\).
\end{task}

\begin{solution}

Для удобства воспользуемся представлением этой суммы в виде диаграммы Юнга. Вот пример представления \(n = 28\):

\begin{figure}[H]
    \centering
    \ytableausetup{centertableaux}
    \begin{ytableau}
        \none[10]   & \none & \none & \none & \none & \none & \none \\
        \none[9]    & \none & \none & \none & \none & \none & \none \\
        \none[8]    &       & \none & \none & \none & \none & \none \\
        \none[7]    &       & \none & \none & \none & \none & \none \\
        \none[6]    &       &       & \none & \none & \none & \none \\
        \none[5]    &       &       &       & \none & \none & \none \\
        \none[4]    &       &       &       &       &       & \none \\
        \none[3]    &       &       &       &       &       & \none \\
        \none[2]    &       &       &       &       &       & \none \\
        \none[1]    &       &       &       &       &       &       \\
        \none       & \none[1] & \none[2] & \none[3] & \none[4] & \none[5] & \none[6]
    \end{ytableau}
    \caption{Диаграмма Юнга: пример разбиения при \(n = 28\).} \label{example_diagram_461}
\end{figure}

На рис.~\ref{example_diagram_461} отмечены \(10\) слагаемых, и в строках значение каждого из них представлено <<кубиками>>, число которых не превышает шести, что соответствует условию задачи. Заметим также, что, если слагаемое равно нулю, <<кубиков>> у него нет и мы считаем, что в разбиении этого слагаемого не существует. Так рис.~\ref{example_diagram_461} можно представить в виде суммы:
\begin{equation*}
    28 = 6 + 5 + 5 + 5 + 3 + 2 + 1 + 1.
\end{equation*}

Указано \(8\) слагаемых, так как два слагаемых в разбиении являются нулями. Условие о том, что каждое разбиение состоит не более, чем из \(10\) слагаемых, также соблюдено.

Представим возможные значения слагаемых <<ящиками>> (от нуля до шести), в которые распределим сами слагаемые~--- <<шары>>. Нам нужна неупорядоченная выборка, в которой значения могут повторяться~--- это число сочетаний с повторениями. Имеем:
\begin{equation*}
    \overline C_{7}^{10} = C_{7 + 10 - 1}^{10} = C_{16}^{10} = \frac{16!}{10!(16 - 10)!} = 8008.
\end{equation*}

Так мы перебрали все представления \(0\leqslant n\leqslant 60\). Однако по условию \(2\leqslant n\leqslant 57\), значит, мы должны исключить варианты:
\begin{enumerate}
    \item \(n = 0\). Его представление единственно: все слагаемые равны нулю, точнее сказать, их вообще нет;
    \item \(n = 1\). Его представление единственно: единственное слагаемое, равное \(1\) (остальные~--- нули);
    \item \(n = 58\). Найдется всего лишь два варианта неупорядоченного разбиения:
        \begin{figure}[H]
            \centering
            \begin{subfigure}[b]{0.45\linewidth}
                \centering
                \ytableausetup{centertableaux}
                \begin{ytableau}
                    \none[10]   &       &       &       &       & \none & \none \\
                    \none[9]    &       &       &       &       &       &       \\
                    \none[8]    &       &       &       &       &       &       \\
                    \none[7]    &       &       &       &       &       &       \\
                    \none[6]    &       &       &       &       &       &       \\
                    \none[5]    &       &       &       &       &       &       \\
                    \none[4]    &       &       &       &       &       &       \\
                    \none[3]    &       &       &       &       &       &       \\
                    \none[2]    &       &       &       &       &       &       \\
                    \none[1]    &       &       &       &       &       &       \\
                    \none       & \none[1] & \none[2] & \none[3] & \none[4] & \none[5] & \none[6]
                \end{ytableau}
                \caption{Диаграмма Юнга: \(n = 58\).} \label{young_diagram_58_1}
            \end{subfigure}
            \begin{subfigure}[b]{0.45\linewidth}
                \centering
                \ytableausetup{centertableaux}
                \begin{ytableau}
                    \none[10]   &       &       &       &       &       & \none \\
                    \none[9]    &       &       &       &       &       & \none \\
                    \none[8]    &       &       &       &       &       &       \\
                    \none[7]    &       &       &       &       &       &       \\
                    \none[6]    &       &       &       &       &       &       \\
                    \none[5]    &       &       &       &       &       &       \\
                    \none[4]    &       &       &       &       &       &       \\
                    \none[3]    &       &       &       &       &       &       \\
                    \none[2]    &       &       &       &       &       &       \\
                    \none[1]    &       &       &       &       &       &       \\
                    \none       & \none[1] & \none[2] & \none[3] & \none[4] & \none[5] & \none[6]
                \end{ytableau}
                \caption{Диаграмма Юнга: \(n = 58\).} \label{young_diagram_58_2}
            \end{subfigure}
            \caption{Диаграмма Юнга: два варианта разбиения при \(n = 58\).} \label{young_diagram_58}
        \end{figure}
    \item \(n = 59\). Его представление единственно: все слагаемые равны \(6\), кроме одного со значением \(5\);
    \item \(n = 60\). Его представление единственно: все слагаемые равны \(6\).
\end{enumerate}

Значит, исключаем \(6\) лишних разбиений. И получаем \textbf{ответ}:
\begin{equation*}
    \overline C_{7}^{10} - 6 = 8008 - 6 = 8002.
\end{equation*}

\end{solution}
