\begin{task}{460}
Выведите явную формулу для числа неупорядоченных разбиений \(p_3(n)\) для произвольного \(n\). Ответ не должен являться рекуррентной формулой, но при этом разрешено оставлять его в виде суммы с переменным (зависящим от \(n\)) числом слагаемых.
\end{task}

\begin{solution}

Для удобства воспользуемся представлением этой суммы в виде диаграммы Юнга. Вот пример представления \(n = 8\):

\begin{figure}[H]
    \centering
    \ytableausetup{centertableaux}
    \begin{ytableau}
        \none[3]    &       & \none & \none & \none & \none & \none \\
        \none[2]    &       &       &       & \none & \none & \none \\
        \none[1]    &       &       &       &       & \none & \none \\
        \none       & \none[1] & \none[2] & \none[3] & \none[4] & \none[5] & \none[6]
    \end{ytableau}
    \caption{Диаграмма Юнга: пример разбиения при \(n = 8\).} \label{example_diagram_460}
\end{figure}

На рис.~\ref{example_diagram_460} отмечены три слагаемых, и в строках значение каждого из них представлено <<кубиками>>, число которых не превышает \(n - 2\). Почему именно такое ограничение на значение слагаемого? По условию задачи их обязательно должно быть три, значит, каждый из них не может быть нулевым, поэтому хотя бы один <<кубик>> у каждого слагаемого должен быть. Остается разместить \(n - 3\) <<кубиков>>. Чтобы найти максимальное значение слагаемого, поместим их все в одну ячейку, допустим, в третью. Тогда значение третьего слагаемого будет равно \(n - 2\). Больше быть не может:

\begin{figure}[H]
    \centering
    \ytableausetup{centertableaux}
    \begin{ytableau}
        \none[3]    &       & \none & \none & \none & \none & \none \\
        \none[2]    &       & \none & \none & \none & \none & \none \\
        \none[1]    &       &       &       &       &       &       \\
        \none       & \none[1] & \none[2] & \none[3] & \none[4] & \none[5] & \none[6]
    \end{ytableau}
\end{figure}

Обозначим \(a_1, a_2, a_3\) за три слагаемых, сумма которых равна \(n\). Попробуем отыскать количество решений уравнения \(a_1 + a_2 + a_3 = n\), рассуждая следующим образом:
\begin{enumerate}
    \item Нужно найти количество неупорядоченных разбиений. Два решения считаются эквивалентными, если они одинаковы с точностью до перестановок слагаемых. Тогда расположим слагаемые в порядке возрастания: \(a_1 \leqslant a_2 \leqslant a_3\);
    
    \item Слагаемых должно быть ровно три, значит, каждое из них ненулевое. Так мы определили нижнюю грань: \(1 \leqslant a_1 \leqslant a_2 \leqslant a_3\);
    
    \item Попробуем отыскать верхние грани для каждого из слагаемых:
    \begin{enumerate}
        \item Заметим, что, если \(a_1 = k\), то \(a_2, a_3 \geqslant k\). Тогда максимально возможное значение \(a_1\)~--- это \(\displaystyle \floor{\frac{n}{3}}\), ведь в противном случае сумма слагаемых превысит \(n\);
        
        \item Зафиксируем \(a_1 = k\). Тогда уравнение примет вид: \(a_2 + a_3 = n - k\). Аналогично рассуждениям предыдущего пункта выводим \(\displaystyle a_2 \leqslant \floor{\frac{n - k}{2}}\);
        
        \item Верхнюю грань \(a_3\) искать нет смысла, потому что оно зависит от выбора подходящих \(a_1, a_2\). Однако это было сделано выше при описании рис.~\ref{example_diagram_460}: \(a_3 \leqslant n - 2\).
    \end{enumerate}
\end{enumerate}

Наш ответ теперь можно явно получить перебором:
\begin{equation*}
    S_{3}(n) = \mathlarger{\sum_{k = 1}^{\floor{\frac{n}{3}}} \sum_{m = k}^{\floor{\frac{n - k}{2}}} 1} =
    \sum_{k = 1}^{\floor{\frac{n}{3}}} \left( \floor{\frac{n - k}{2}} - k + 1 \right).
\end{equation*}

\textbf{Ответ:} \(\displaystyle S_{3}(n) = \sum_{k = 1}^{\floor{\frac{n}{3}}} \left( \floor{\frac{n - k}{2}} - k + 1 \right)\).

\end{solution}
