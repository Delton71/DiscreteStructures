\begin{task}{455}
Сколькими способами можно расставить числа от \(1\) до \(15\) по кругу так, чтобы никакая пятерка чисел с одинаковым остатком по модулю \(3\) не стояла рядом (четыре числа с одинаковым остатком, соседствующие друг с другом, допустимы, а вот все пять — нет)? При этом расстановки считайте одинаковыми, если они получаются друг из друга циклическими сдвигами.
\end{task}

\begin{solution}

\begin{enumerate}
    \item Найдем общее число перестановок $\displaystyle K = \frac{15!}{15} = 14!$, так как циклические сдвиги считаются одинаковыми.
    
    \item Пусть $S_i$ --- количество перестановок, в которых числа с остатком $i$ стоят рядом и образуют пятерку. Найдем $K_1, K_2, K_3$:
    \begin{align*}
        K_1 &= \sum_{i=0}^{2} |S_i| = C_3^1 \cdot 5! \cdot 10!, \\
        K_2 &= \sum_{i<j} |S_i \cap S_j| = C_3^2 \cdot 5!^2 \cdot 6 \cdot 5!, \\
        K_3 &= |S_0 \cap S_1 \cap S_2| = C_3^3 \cdot 5!^3 \cdot 2.
    \end{align*}
    
    Здесь число сочетаний определяет, по какому модулю мы выбрали пятерки чисел; $(5!)^i$~--- перестановки внутри пятерок. Оставшиеся слагаемые определены так:
    \begin{enumerate}
        \item \textbf{Случай $K_1$:} $10!$ --- это перестановки остальных элементов;
        
        \item \textbf{Случай $K_2$:} \(6\) отвечает за варианты положения второй пятерки относительно первой, $5!$ --- за перестановки остальных элементов;
        
        \item \textbf{Случай $K_3$:} \(2\) отвечает за уникальные комбинации пятерок. Всего их может быть шесть (числами обозначены общие для пятерок остатки при делении на три):
        \begin{equation*}
            (0, 1, 2), (0, 2, 1), (1, 0, 2), (1, 2, 0), (2, 0, 1), (2, 1, 0),
        \end{equation*}
        из которых остается только две уникальные: $(0,~1,~2), (0,~2,~1)$.
    \end{enumerate}
    
    По формуле включений-исключений получаем:
    \begin{multline*}
        |K \backslash (S_0 \cup S_1 \cup S_2)| = |K| - |K_1| + |K_2| - |K_3| = \\
        = 14! - C_3^1 \cdot 5! \cdot 10! + C_3^2 \cdot 5!^2 \cdot 6 \cdot 5! - C_3^3 \cdot 5!^3 \cdot 2 = \\
        = 14! - 5!^2 \cdot \left( 3 \cdot \left( 6 \cdot 7 \cdot 8 \cdot 9 \cdot 10 \right) - 3 \cdot 6 \cdot 5! + 5! \cdot 2 \right) = \\
        = 14! - 5!^3 \cdot (6 \cdot 7 \cdot 9 \cdot 2 - 3 \cdot 6 + 2) = 14! - 5!^3 \cdot \left( 756 - 18 + 2 \right) = \\
        = 14! - 5!^3 \cdot 740 = 14! - 5!^3 \cdot 4 \cdot 5 \cdot 37 = 85899571200.
    \end{multline*}
\end{enumerate}

\textbf{Ответ:} 85899571200.

\end{solution}
