\begin{task}{191}
Дана последовательность \(a_n\), заданная линейным рекуррентным соотношением \(a_{n+3}+a_{n+2}-16a_{n+1}+20a_n=0\) при условии, что \(a_0=4, a_1=2, a_2=20\). Вычислите значение выражения \(\sum_{n=0}^{+\infty}na_n\left(\frac{1}{10}\right)^n\), используя <<свёртку>> производящей функции.
\end{task}

\begin{solution}

Найдем <<свёртку>> производящей функции:
\begin{gather*}
    G(z) = a_0 + a_1z + a_2z^2 + \sum_{n=3}^{\infty} \left(-a_{n-1} + 16a_{n-2} - 20a_{n-3} \right)z^n = \\
    = a_0 + a_1z + a_2z^2 - z\sum_{n=2}^{\infty} a_nz^n + 16z^2\sum_{n=1}^{\infty} a_nz^n - 20z^3\sum_{n=0}^{\infty} a_nz^n = \\
    = a_0 + a_1z + a_2z^2 - z\left(G(z) - a_0 - a_1z \right) + 16z^2\left(G(z) - a_0 \right) - 20z^3G(z) = \\
    = 4 + 2z + 20z^2 - z\left(G(z) - 4 - 2z \right) + 16z^2\left(G(z) - 4 \right) - 20z^3G(z) = \\
    = G(z)(-z + 16z^2 - 20z^3) + 4 + z\left(2 + 4 \right) + z^2\left(20 + 2 - 16 \cdot 4 \right) = \\
    = G(z)(-z + 16z^2 - 20z^3) + 4 + 6z - 42z^2.
\end{gather*}

В итоге находим \(G(z) = \frac{4 + 6z - 42z^2}{1 + z - 16z^2 + 20z^3}\).

Если ряд представить суммой, то:
\begin{equation*}
    G(z) = \sum_{n=0}^{\infty} a_nz^n \Rightarrow \sum_{n=0}^{\infty} na_nz^n = z \cdot G'(z).
\end{equation*}

Обозначим \(P(z) = z \cdot G'(z)\). Тогда \(P(0.1)\) даст требуемый ответ. Однако следует проверить сходимость \(P(z)\) при \(z = 0.1\). Если радиус сходимости этого ряда, соответствующий в нашем случае наименьшему по модулю корню знаменателя \(G(z)\), окажется больше \(z = 0.1\), тогда ряд сходится. В таком случае нам нужно удостовериться, что:
\begin{equation*}
    \left(\forall x \in \left[-\frac{1}{10}; \frac{1}{10} \right] \right) T(x) = 1 + x - 16x^2 + 20x^3 \neq 0.
\end{equation*}

Найдем производную \(T(x)\), её корни, отсюда границы возрастания и убывания самой функции и определим, есть ли нули у \(T(x)\) на отрезке \([-0.1; 0.1]\):
\begin{equation*}
    T'(x) = 60x^2 - 32x^2 + 1 = (30x - 1)(2x - 1); \: T(\frac{1}{30}) = \frac{254}{250}.
\end{equation*}

Отсюда:
\begin{align*}
    \text{Возрастает на } [-\frac{1}{10}; \: &\frac{1}{30}]; T(-\frac{1}{10}) = \frac{18}{25}, \\
    \text{Убывает на } [\frac{1}{30}; \: &\frac{1}{10}]; T(\frac{1}{10}) = \frac{24}{25}.
\end{align*}

Исходя из промежутков возрастания и убывания функции, а также её значений в крайних точках \([-0.1; 0.1]\), делаем вывод, что на отрезке функция \(T(x)\) никогда не обращается в ноль. Следовательно, радиус сходимости больше \(0.1\), и ряд \(P(x)\) в этой точке будет сходится. Найдем его значение:
\begin{gather*}
    P(z) = z \cdot G'(z) = z \cdot \frac{420z^3 + 90z^2 - 48z - 2}{(1 + z - 16z^2 + 20z^3)^2}; \\
    P(\frac{1}{10}) = \frac{1}{10} \cdot \frac{-5.48}{0.9216} = -\frac{685}{1152}.
\end{gather*}

\textbf{Ответ:} \(\displaystyle -\frac{685}{1152}\).

\end{solution}
