\begin{task}{401}
Пусть дана последовательность различных натуральных чисел длины \(k^4+1\). \textbf{Используя теорему Мирского или теорему Дилуорса}, докажите, что в этой последовательности можно выделить либо подпоследовательность длины \(k+1\), в которой каждое следующее число \emph{делится} на предыдущее, либо подпоследовательность длины \(k+1\), в которой каждое следующее число \emph{делит} предыдущее, либо монотонную подпоследовательность длины \(k+1\), в которой любые два числа не являются делителями друг друга. \textbf{При решении нужно обязательно оформить формулировку используемой теоремы с помощью окружения \emph{theorem}, а также сослаться на неё при помощи команды \emph{ref}}.
\end{task}

\begin{solution}

\begin{theorem} [Теорема Дилуорса] \label{Dilworth}
    Наименьшее количество цепей, покрывающих конечное ЧУМ, равно наибольшему размеру антицепи в этом ЧУМе.
\end{theorem}

Зададим на последовательности длины \(k^4+1\) строгий частичный порядок \(\prec\):
\begin{equation*}
    \left( a_i \prec a_j \right) \equiv \left( a_i < a_j \land i < j \right).
\end{equation*}
Это отношение порядка позволяет установить, образуют ли выбранные элементы монотонную подпоследовательность. В том числе, если \(a \prec b\), тогда \(\left( a, b \right)\)~--- монотонно возрастающая подпоследовательность. В противном случае это монотонно убывающая подпоследовательность, так как числа последовательности различные.

Заметим также, что элементы, образующие цепь, также формируют монотонно возрастающую последовательность. Аналогично антицепь является монотонно убывающей последовательностью.

Так в последовательности \(\left( 1, 8, 2, 6, 5, 4, 7 \right)\) можно выделить монотонно возрастающую подпоследовательность \(\left( 1, 2, 4, 7 \right)\), а также монотонно убывающую подпоследовательность \(\left( 8, 5, 4 \right)\).

Вместе с порядком \(\prec\) последовательность длины \(k^4+1\) становится ЧУМ, и теперь можно применить Теорему Дилуорса~\ref{Dilworth}:
\begin{enumerate}
    \item Если нашлась антицепь размера \(k^2+1\), тогда мы нашли монотонно \emph{убывающую} подпоследовательность;
    
    \item Если же наибольший размер антицепи меньше \(k^2+1\), значит, можно найти покрытие ЧУМа размера \(k^4+1\) не более чем \(k^2\) цепями, и, используя метод Дирихле, делаем вывод, что среди них обязательно найдется цепь длины не меньше \(k^2+1\)~--- то есть мы нашли монотонно \emph{возрастающую} подпоследовательность.
\end{enumerate}

В любом случае выявится монотонная подпоследовательность длины \(k^2+1\). Снова применяем теорему Дилуорса~\ref{Dilworth} для ЧУМа, получившегося из новой теперь уже \emph{монотонной} последовательности длины \(k^2+1\), но с новым порядком делимости |:
\begin{enumerate}
    \item Если наибольший размер антицепи меньше \(k+1\), значит, можно найти покрытие ЧУМа размера \(k^2+1\) не более чем \(k\) цепями, и, используя метод Дирихле, делаем вывод, что среди них обязательно найдется цепь длины не меньше \(k+1\). Тогда:
    \begin{enumerate}
        \item Если нашлась цепь размера \(k+1\) в \emph{монотонно убывающей} последовательности длины \(k^2+1\), тогда мы нашли искомую \emph{подпоследовательность длины \(k+1\), в которой каждое следующее число делится на предыдущее};

        \item Если нашлась цепь размера \(k+1\) в \emph{монотонно возрастающей} последовательности длины \(k^2+1\), тогда мы нашли искомую \emph{подпоследовательность длины \(k+1\), в которой каждое следующее число \emph{делится} на предыдущее};
    \end{enumerate}
    
    \item В противном случае найдется антицепь размера \(k+1\) в \emph{монотонной} последовательности длины \(k^2+1\), значит, мы нашли искомую \emph{монотонную подпоследовательность длины \(k+1\), в которой любые два числа не являются делителями друг друга}.
\end{enumerate}

Таким образом, условие задачи выполнено.

\end{solution}