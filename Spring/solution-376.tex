\begin{task}{376}
Целое число \(x\) называется \emph{квадратичным вычетом} по модулю \(m\), если \((x, m)=1\) и существует такое целое \(y\), что \(x \equiv y^2 \pmod{m}\). Пусть даны попарно взаимно простые числа \(m_1,m_2,\dots,m_n\in \mathbb{N}\) и \(M=\prod_{i=1}^n m_i\). Используя китайскую теорему об остатках (необязательно использовать явный вид решения, достаточно существования и единственности), докажите, что \(x\) является квадратичным вычетом по модулю \(M\) тогда и только тогда, когда \(x\) является таковым по каждому модулю \(m_i\). 

\textbf{Техническое требование:} при оформлении запишите формулировку теоремы, которой будете пользоваться, с помощью окружения \emph{theorem}.
\end{task}

\begin{solution}

Ниже приведена формулировка Китайской теоремы об остатках (КТО):
\begin{theorem}\label{theorem_CRT}
Если натуральные числа \(m_{1}, m_{2}, \dots, m_{n}\) попарно взаимно просты, то для любых целых \(r_{1}, r_{2}, \dots, r_{n}\) таких, что \(0 \leqslant r_{i} < m_{i}\) при всех \(i \in \{1, 2, \dots, n\}\), найдётся число \(M\), которое при делении на \(m_{i}\) даёт остаток \(r_{i}\) при всех \(i \in \{1, 2, \dots, n\}\). Более того, если найдутся два таких числа \(M_1\) и \(M_2\) (соответствующих утверждению выше), то \(M_{1} \equiv M_{2}{\pmod  {m_{1} \cdot m_{2} \cdot \ldots \cdot m_{n}}}\).
\end{theorem}

\textbf{Докажем в правую сторону \((\Rightarrow)\)}.

Если \(x\) является квадратичным вычетом \(M\), то \(x\) также является квадратичным вычетом \(m_i \left(1 \leqslant i \leqslant n \right)\), так как каждый \(m_i \: | \: M\).

\textbf{Докажем в обратную сторону \((\Leftarrow)\)}.

Пусть \(x \equiv y_i^2 \pmod{m_i} \left(1 \leqslant i \leqslant n \right)\). Используя КТО (Теорема~\ref{theorem_CRT}), обнаружим:
\begin{equation*}
    \exists! \: K: \: \left(1 \leqslant i \leqslant n \right) \left[ K \equiv y_i \pmod{m_i} \Rightarrow K^2 \equiv y_i^2 \equiv x \pmod{m_i} \right]. 
\end{equation*}

Используя ту же КТО (Теорема~\ref{theorem_CRT}), точнее, часть о единственности, находим, что \(K^2 \equiv y_i^2 \equiv x \pmod{m_i} \Rightarrow x \equiv K^2 \pmod{M}\).

Осталось проверить, что выполняется условие:
\begin{equation*}
    \left[ \forall \left(1 \leqslant i \leqslant n \right) (x, m_i) = 1 \right] \Leftrightarrow (x, M) = 1.
\end{equation*}

Оно, очевидно, справедливо ввиду взаимной простоты всех \(m_i\) между собой и с числом \(x\).

\end{solution}
