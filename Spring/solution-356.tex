\begin{task}{356}
При каких \(k,n,t\) будет заведомо существовать система различных представителей для совокупности из \(t\) различных \(k\)-элементных подмножеств \(n\)-элементного множества? А при каких \(k,n,t\) с.\,р.\,п. заведомо не будет существовать?
\end{task}

\begin{solution}

При решении задачи будем пользоваться следствием теоремы Холла:
\begin{corollary}\label{corollary_Hall}
Система различных представителей для семейства подмножеств \(\{A_1, \ldots, A_s\}\) существует тогда и только тогда, когда объединение \(m\::\:\left( 1 \leqslant m \leqslant s \right)\) множеств \(A_i\) содержит не менее \(m\) элементов.
\end{corollary}

Во-первых, заметим, что \(C_{k + 1}^{k} = k + 1\), то есть объединение \(k + 1\) различных \(k\)-элементных подмножеств будет содержать по крайней мере \(k + 1\) элемент. По следствию~\ref{corollary_Hall} система различных представителей будет существовать. Этим мы выделяем \emph{существование с.\,р.\,п. при \(t = k + 1, n \geqslant k + 1\)}. Каждое подмножество \(k\)-элементное, \(k \leqslant |\cup A_i|\), значит, для любого \(1 \leqslant m \leqslant s = k\), \emph{то есть при \(t < k + 1\), с.\,р.\,п. также найдется}.

Приведем пример с условием \(t > k + 1\), где с.\,р.\,п. не будет существовать.

Пусть \(n \geqslant k + 2, s = k + 2\). Возьмем \(k + 1\) различных \(k\)-элементных подмножеств, объединение которых будет содержать ровно \(k + 1\) элемент. Добавим к ним еще одно множество, в котором будет присутствовать \(k + 2\) элемент, не содержащийся в остальных \(k + 1\) подмножествах. Тогда по следствию~\ref{corollary_Hall} при \(1 \leqslant m = k + 1 \leqslant s = k + 2\) объединение первых \(k + 1\) подмножеств должно содержать не менее \(k + 2\) элементов, однако это не так. Данное противоречие позволяет нам утверждать, что \emph{при \(t > k + 1, n \geqslant k + 2\) с.\,р.\,п. не существует}.

Так мы определили, что \textbf{при \(\left(t \leqslant k + 1, n \geqslant t\right)\) с.\,р.\,п. всегда найдется}.

Дополнительно в рассуждениях мы всегда отмечали, что \(n \geqslant t\). Действительно, если \(n < t\), тогда объединение \(t\) различных \(k\)-элементных подмножеств таково, что его мощность обязательно будет не больше \(n\), значит, вместе с тем меньше \(t\), что противоречит следствию~\ref{corollary_Hall}:
\begin{equation*}
    t = |\cup A_i| \leqslant n < t.
\end{equation*}

Таким образом:
\begin{enumerate}
    \item с.\,р.\,п. заведомо существует при \(t \leqslant k + 1, n \geqslant t\);
    
    \item с.\,р.\,п. заведомо не будет существовать при \(n < t\).
\end{enumerate}

\end{solution}
